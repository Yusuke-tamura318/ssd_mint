% !TeX root = paper.tex


\chapter{序論}

電車の車両タイプは JR の在来線だけでも 100 種類近く存在している.
多くの人は電車を見て電車だと認識することは可能だが,その電車の車両タイプまでを判断できる人は少ない.
電車についての知識がある人は一目見るだけでその電車の車両タイプを判断できるが,大多数の人は似ている電車の車両タイプを判断することが難しい.
現在,車両タイプを調べる方法としてGoogle レンズを使用することや,図鑑と見比べる方法が挙げられる.
googleレンズは一枚の画像に2つ以上の電車が写っていると車両タイプを正確に分類できず,分類結果は車両タイプが出力されるのではなく,入力画像に写っているモノと似ているものが写っているウェブページ一覧を出力するので,出力されたウェブサイトを開いて知りたい情報を自分で探し出す必要がある.
図鑑と見比べる方法では知りたい情報を得るために多くの時間が必要になる.

現在の方法では画像に写っている車両タイプが何なのかが知りたいときに余計な手間をかける必要があるという問題がある.
本プロジェクトではこの問題を解決するために画像や動画に写っている車両が何なのかを判別できるシステムを開発する.

%序論には研究の背景と目的を書く.どういう問題があって,現在はどういう手法がとられているなど,先行研究で試みられている手法などを参考文献を交えながら書く.序論の終わりには,各章に何を述べたかを簡潔に記述する.
%現在,似ている電車がいっぱいある
%それぞれがどの車両タイプの電車なのかを判断したいよ
%画像を読み込ませることで,その画像には何が写っているのか判断するシステムを開発する\\
%画像に写っている電車が何なのかを判断するためには電車の図鑑と画像を見比べて,自分で判断する方法しかない.
%その車両が何なのか知るために大きな労力が必要であることが問題である.\\
%現在の画像分類では,画像に写っているのが,人や車,犬など,大まかな分類しかすることができない.
%この方法では電車の車両タイプが何かを判断することができない.\\
%本プロジェクトでは,電車が写った画像を与えることで車両タイプを判別するシステムを開発する.


