% !TeX root = paper.tex


\chapter{序論}


%序論には研究の背景と目的を書く.どういう問題があって,現在はどういう手法がとられているなど,先行研究で試みられている手法などを参考文献を交えながら書く.序論の終わりには,各章に何を述べたかを簡潔に記述する.
(背景)\\
現在,似ている電車がいっぱいある

(目的)\\
それぞれがどの車両タイプの電車なのかを判断したいよ
画像を読み込ませることで,その画像には何が写っているのか判断するシステムを開発する\\
(問題点)\\
画像に写っている電車が何なのかを判断するためには電車の図鑑と画像を見比べて,自分で判断する方法しかない.
その車両が何なのか知るために大きな労力が必要であることが問題である.\\
(現在の手法)\\
現在の画像分類では,画像に写っているのが,人や車,犬など,大まかな分類しかすることができない.
この方法では電車の車両タイプが何かを判断することができない.\\