% !TeX root = paper.tex


\appendix %付録
\chapter{開発したプログラム}
\section{セットアップ方法}


にプログラムの使い方,セットアップ方法などを書きましょう.ここにプログラここにプログラムの使い方,セットアップ方法などを書きましょう.ムの使い方,セットアップ方法などを書きましょう.
ここにプログラムの使い方,セットアップ方法などを書きましょう.

\begin{Verbatim}[frame=single]
$ sudo apt-get install python3-pip      # PIPコマンドの導入
$ echo "Hello WOrld"
Hello WOrld
\end{Verbatim}

にプログラムの使い方,セットアップ方法などを書きましょう.ここにプログラここにプログラムの使い方,セットアップ方法などを書きましょう.ムの使い方,セットアップ方法などを書きましょう.
ここにプログラムの使い方,セットアップ方法などを書きましょう.

\section{使い方}
ここにプログラムの使い方,セットアップ方法などを書きましょう.
ここにプログラムの使い方,セットアップ方法などを書きましょう.ここにプログラここにプログラムの使い方,セットアップ方法などを書きましょう.ムの使い方,セットアップ方法などを書きましょう.
ここにプログラムの使い方,セットアップ方法などを書きましょう.ここにプログラムの使い方,セットアップ方法などを書きましょう.
%%%%%%%%%%%%% プログラムの埋め込み %%%%%%%%%%%%%%%%%%%%%%%%%

\section{ソースコード}
%% ファイル名を指定して、挿入する場合
%\lstinputlisting[language=c,caption=サンプルプログラム,label=sample.c]{appendix/src/sample.c}
%project_processing.pyとsave.py
\lstinputlisting[language=Python,caption=画像の保存1,label=project_processing.py]{appendix/src/project_processing.py}
\lstinputlisting[language=Python,caption=画像の保存2,label=save.py]{appendix/src/save.py}


\chapter{いいいいい}
あああああああああああああああああああああいいいいいいいいいいいいいいいいいいいいいいいいいいいいううううううううううううううううう
