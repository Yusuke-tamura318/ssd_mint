% !TeX root = paper.tex


\begin{center}
		{\gtfamily \Large 概要}
\end{center}
\vspace{1cm}
\thispagestyle{empty}	
%%%%%%%%%%%%%%%%%%%%%%%%%%%%%%% 概要ここから



%概要には目的,背景,手法,実験方法,実験結果を簡潔にまとめたものを書く(つまり,論文を1ページに圧縮したものである.概要だけを読%んでも,大まかに何をやって,どのような結果になったかが分かるように書くこと.0.5ページ分くらいはうめること.).○○○○ 
本プロジェクトでは,機械学習を用いた電車の車両タイプを分類するシステムの開発を行った.

世の中には似ているようなものでも,実は同じではないものがある.
%そのようなものを判断できるようになりたい.
電車の車両タイプの種類は,JRの在来線だけでも100種類近く存在している.
多くの人は電車を見て,電車だと認識することは可能であるが,その電車の車両タイプまでを判断できる人は少ない.
電車についての知識がある人は一目見るだけでその電車の車両タイプを判断できるが,大多数の人は似ている電車の車両タイプを判断することが難しい.
そのため,だれでも簡単に電車の車両タイプを判断できるシステムの開発を行った.
今回はJR西日本の17種類の電車を分類,識別するシステムの開発を行った.

モデルの開発は,画像を集めてデータセットを作成し,YOLOとデータセット用いて学習させ評価を行う.
学習データは車両タイプ別にYouTubeにアップロードされている動画から電車が写っている場面を切り出した画像を利用した.
一つの動画から一種類の車両タイプのデータセットを作成すると,似たような画像が大量に保存されてしまうため,複数の動画の任意の場面を連結して一本の動画にするシステムをPythonのフレームワークであるDjangoを利用して作成し,様々な場面の電車の画像が保存できるようにした.
%学習を進めると徐々に性能が向上していき,学習を続けても性能が向上しなくなると学習は中断される.学習は中断されるまで続けてモデルを作成した.
学習を進めていき性能が向上しなくなるまで学習をさせてモデルを作成した.
外見が似ている車両タイプの判別の間違いが多かった.判別対象の車両タイプのなかで外見に特徴がある車両タイプは正確に判別をすることに成功した.

==================ここから田村====================\\
作成したモデルを使ってブラウザ上で電車の車両タイプを分類するWEBアプリも作成した.\\

%JR北(5)JR東(33)JR東海(11)JR西(25)JR四国(5)JR九州(15)
%%%%%%%%%%%%%%%%%%%%%%%%%%%%%%%%% 概要ここまで
\clearpage
